\documentclass[11pt,a4paper]{scrreprt}
%Umlaute ohne Codierung
\usepackage[utf8]{inputenc}
%europäische Zeichen
\usepackage[T1]{fontenc}
%neue Rechtschreibung
\usepackage[ngerman]{babel}
\usepackage{scrpage2}
\pagestyle{scrheadings}
\clearscrheadfoot

%kein Seitenvorschub
\makeatletter
\newcommand\Chapter{%
                    \par\vspace{0.01cm}% anpassen
                    \global\@topnum\z@
                    \@afterindentfalse
                    \secdef\@chapter\@schapter}
\makeatother

\begin{document}
\begin{titlepage}
	\centering
	{\scshape\LARGE Universität Leipzig \par}
	\vspace{1cm}
	{\scshape\Large Softwaretechnik-Praktikum \par}
	\vspace{2cm}
	{\huge\bfseries Releaseplan Gruppe cz17a - Gamification \par}
	\vspace{2cm}
	{\Large\itshape Lisa Vogelsberg, Felix Fink, Michael Fritz, Thomas Gerbert, Steven Lehmann, Fabian Ziegner, Willy Steinbach, 	Christian Schlecht \par}
	\vfill
	supervised by \par
	Dr.~Christian \textsc{Zinke}, Julia \textsc{Friedrich}, Christian \textsc{Frommert}
	\vfill
	{\large \today \par}
\end{titlepage}
\tableofcontents

\Chapter{Arbeitpaket 1 - Vorprojekt}
\textit{Release: 15.01.18 / 22.01.18} \\
Das Vorprojekt bildet das Grundgerüst der Anwendung. Es wird eine Server-Client Architektur entwickelt. Zusätzlich werden eine PostgreSQL Datenbank und eine REST-Schnittstelle erstellt und über den Praktikumsserver angebunden.

\section{Datenbank}
\begin{itemize}
\item Die Datenbank wird modelliert, erstellt und auf den Praktikumsserver geladen.
\item Aus der Datenbank werden die Fragen ausgelesen.
\end{itemize}

\section{Server}
\begin{itemize}
\item der Server übernimmt die zentrale Verarbeitung des Quiz
\item der Server händelt alle Verbindungen zu den Clients, im Rahmen des Vorprojekts beschränkt sich dies auf einen Client
\item über eine REST-Schnittstelle werden alle Anfragen an die Datenbank gestellt
\end{itemize}

\section{Client}
\begin{itemize}
\item der Client erhält vom Server eine Frage inkl. der Antworten
\item das User-Interface ermöglicht die Darstellung der Fragen und Antworten
\item der User kann eine Frage auswählen und erhält eine visuelle Rückmeldung, ob die Frage richtig beantwortet wurde.
\end{itemize}

\section{Admin-Panel}
\begin{itemize}
\item das Admin-Panel wird mittels einer Weboberfläche realisiert.
\item es können Fragen inkl. Antworten im CSV-Format in die Datenbank eingelesen werden.
\end{itemize}


\Chapter{Arbeitspaket 2 - Spieler und Quiz}
\textit{Release: 19.02.2018}

\Chapter{Arbeitspaket 3 - Quizfinalisierung}
\textit{Release: 12.03.18 / 19.03.18}

\Chapter{Arbeitspaket 4 - Administration}
\textit{Release: 02.04.2018}

\Chapter{Arbeitspaket 5 - Dashboard}
\textit{Release: 16.04.2018}

\end{document}