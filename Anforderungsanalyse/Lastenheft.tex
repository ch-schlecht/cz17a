%scrrprt ist eine Klasse für Berichte und längere Arbeiten
\documentclass[11pt,a4paper]{scrreprt}
%ermöglicht die Eingabe von Umlauten usw. ohne Codierung
\usepackage[utf8]{inputenc}
%Schriften werden mit einer passenden Kodierung für europäische Zeichen ausgegeben
\usepackage[T1]{fontenc}
%Passt Dokumentelemente an die Konventionen der deutschen Sprache (neue Rechtschreibung) an, z.b. Datumsangaben, Silbentrennung
\usepackage[ngerman]{babel}
\title{Lastenheft}
\author{Lisa Vogelsberg}
\date{29.11.2017}

\begin{document}
\tableofcontents

%Bei reports beginnt die Zählung von sections mit 0. Daher wurde hier chapter gewählt. Danach kann section gewählt werden, danach subsection und subsubsection
\chapter{Ausgangssituation}
%LV: Übernommen aus der Themenbeschreibung im OLAT
Wissensspiele erfreuen sich im privaten Sektor höchster Popularität (Wer wird Millionär, etc.). Im Bereich der Weiterbildung und des Wissensaustausches in Unternehmen werden diese bisher jedoch nur selten eingesetzt. Als Erweiterung zu klassischen E-Learning-Anwendungen haben Wissensspiele ein erhebliches Potential. Im Rahmen des aktuellen Projektes SB:Digital (http://sbdigital.infai.org/) werden neue Formen des gamifizierten Lernens und des Wissensaustausches entwickelt und erprobt.

\chapter{Zielsetzung und Produkteinsatz}
\section{Vision}
%LV: Ich krieg das nicht schön bzw. prägnant ausformuliert, deswegen erstmal so. Vielleicht kann das jemand besser ausdrücken.
Mitarbeiter von Unternehmen sollen mit Spaß und Unterhaltung dazu motiviert werden, sich an der betrieblichen Weiterbildung zu beteiligen. Wissen soll vermittelt werden, ohne dass es als Last empfunden wird. Im besten Fall wollen die Mitarbeiter sich an der Weiterbildung über das gegebene notwendige Maß hinaus beteiligen.
\section{Zielsetzung}
Die zu entwickelnde Software soll unter Verwendung verschiedener Techniken der Gamification die betriebliche Weiterbildung und den Wissensaustausch unterstützen.
Als Kernkomponente soll ein Quiz mit einer Frage, vier Auswahlmöglichkeiten und einer richtigen Antwort erstellt werden, an dem Gruppen von mindestens 5 Personen teilnehmen.
Gewinnen soll dabei aber nicht immer nur derjenige, der am meisten weiß, sondern derjenige, der Wissen, Strategie und Zufall für sich zu nutzen weiß.
\section{Produkteinsatz}
Zielgruppe des Produkts sind Arbeitnehmer in Unternehmen verschiedener Branchen, die sich während ihrer Arbeitszeit mit dem Quiz beschäftigen. Die Teilnahme erfolgt anonym dahingehend, dass nicht festgestellt werden kann, welche reale Person gerade an dem Quiz teilnimmt.

\chapter{Funktionale Anforderungen}
%LV: Muss- und Kann-Ziele finde ich für die Anforderungen noch nicht ganz passend, da man sie zu schnell mit den Zielen verwechseln die kann, die man vorher festlegt hat und durch die sich dann die Anforderungen ergeben. Einteilung in Hauptanforderungen und optionale Anforderungen trifft es vielleicht schon eher, ist trotzdem noch nicht gut von mir gewählt (Hauptanforderungen meint nämlich eigentlich etwas anderes). Würde ich Dienstag gern besprechen, wenn wir bis dahin nicht noch eine bessere Bezeichnung gefunden haben.
\section{Hauptanforderungen}
%Erzeugt eine Liste ohne Nummerierung, bei dem das Wort in eckigen Klammern als Beschreibung dient und fett gedruckt wird. Neue Zeilen innerhalb eines items werden eingerückt.
\subsection{An- und Abmelden}
\begin{description}
\item[/LF0010/] Registrierung \\
Ein Nutzer soll sich unter Angabe eines Nutzernamens, Passworts und E-Mail registrieren können.
\item[/LF0020/] Anmelden \\
Ein Nutzer soll sich mit seinem Nickname und Passwort einloggen können.
\item[/LF0030/] Abmelden \\
Der Nutzer soll sich vom System abmelden können.
\item[/LF0040/] Passwort anfordern \\
Der Nutzer soll sein Passwort anfordern können, wenn er dieses vergessen hat.
\end{description}
\subsection{Quiz}
\begin{description}
\item[/LF0010/] Fragenpool \\
Es muss ein Pool an Fragen zu einem Thema vorhanden ein, aus dem für das Quiz Fragen entnommen werden können.
\item[/LF0020/] \ \\
Jede Quizrunde soll dynamisch aus * Fragen zusammengestellt werden.
\item[/LF0030/] Antwort auswählen \\
Der Spieler soll aus vier Antwortmöglichkeiten genau eine auswählen können, die richtig ist.
%LV: Die nächste Anforderung muss weitergehend spezifiziert werden bzw. um weitere Anforderungen erweitert werden, was einem zusätzlich noch angezeigt wird oder wann und wie sich ein Score verändert.
\item[/LF0040/] \ \\ 
Nach Auswahl einer Antwort soll die Anwendung eine Rückmeldung darüber geben, ob die richtige Antwort gewählt wurde. %LV: Vielleicht auch, was die richtige Antwort war, wenn man falsch lag?
\item[/LF0050/] Ergebnisse ausgeben \\
Am Ende einer Quizrunde soll eine zusammenfassende Ausgabe der Ergebnisse erfolgen.
\end{description}
\begin{description}
\item[/LF0010/] Wissenspiel für 5 Spieler
\item[/LF0040/] Fragen und spielerbezogene Daten müssen gespeichert werden können
\item[/LF0050/] ACL zum Administrieren
\item[/LF0060/] Erstellen von Wissensspielen
\end{description}
%CS: denkt ihr wir sollten alle Gamification Elemente direkt als Muss-Ziel verankern? was wenn sich ergibt, dass es nicht/schlecht umsetzbar ist
Gamification Elemente:
\begin{description}
\item[/LF0070/] Badges
\item[/LF0080/] Ranking-System/Leaderboards 
\item[/LF0080/] Ein Leaderboard soll die besten * Mitspieler bzw. nur deren Punkte anzeigen.
\item[/LF0090/] Ratingsystem für Fragen und Antworten
\item[/LF0100/] Freischalten von Inhalten
\end{description}
\subsection{Dashboard}
\begin{description}
\item[/LF0010/] Die Auswertung von Spielergebnissen soll über ein Dashboard erfolgen
\end{description}
\section{Optionale Anforderungen}
\subsection{Spielerprofil}
\begin{description}
\item[/LF0010/] \ \\ 
Für jeden Benutzer soll ein eigenes Profil anwählbar sein bestehend aus
	\begin{itemize}
	\item Nutzername
	\item Profilbild (optional)
	\item Kuze Nachricht (optional)
	\end{itemize}
\end{description}
\chapter{Nicht-funktionale Anforderungen}
\section{Hauptanforderungen}
\begin{description}
\item[/LL0010/]
\item[/LL0020/]
\item[/LL00XX/]
\end{description}
\section{Optionale Anforderungen}
\begin{description}
\item[/LL0010/]
\item[/LL0020/]
\item[/LL00XX/]
\end{description}

\chapter{Qualitätsmatrix nach ISO 25010}
% Erzeugt eine Tabelle. Die zweiten Parameter geben allgemeine Einstellungen für die Spalten an. l:linksbündig, c:center und r:rechtsbündig. Mit & wird eine Spalte erstellt, \\ erstellt eine neue Zeile. \hline erzeugt eine horizontale Linie.
\begin{tabular}{|l|c|c|c|c|}
\hline
		& hoch & mittel & niedrig& nicht anwendbar\\
\hline
Funktionalität  &              &              & 		&\\     
Zuverlässigkeit	&              &              & 		&\\
Effizienz 		&              &              & 		&\\
Sicherheit  	&              &              & 		&\\
Kompatibilität  &              &              & 		&\\
Benutzbarkeit  	&              &              & 		&\\
Wartbarkeit  	&              &              & 		&\\
Portierbarkeit  &              &              & 		&\\
\hline
\end{tabular}

\chapter{Lieferumfang und Abnahmekriterium}
\begin{itemize}
\item
\end{itemize}

\chapter{Vorprojekt}
\begin{itemize}
%CS: to discuss
\item Mögliches Vorprojekt: \\
Im Rahmen des Vorprojekt soll das wesentliche Grundgerüst, bestehend aus Datenbank, Server - und Client-Grundkonzept und Fragenmechanismus, erstellt werden. Es soll bereits möglich sein, die Fragen aus der Datenbank abzufragen, anzuzeigen und evtl. bereits zu beantworten.
\end{itemize}

\chapter{Glossar}
\begin{itemize}
\item
\end{itemize}
\end{document}