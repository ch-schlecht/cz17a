%scrrprt ist eine Klasse für Berichte und längere Arbeiten
\documentclass[11pt,a4paper]{scrreprt}
%ermöglicht die Eingabe von Umlauten usw. ohne Codierung
\usepackage[utf8]{inputenc}
%Schriften werden mit einer passenden Kodierung für europäische Zeichen ausgegeben
\usepackage[T1]{fontenc}
%Passt Dokumentelemente an die Konventionen der deutschen Sprache (neue Rechtschreibung) an, z.b. Datumsangaben, Silbentrennung
\usepackage[ngerman]{babel}
\title{Lastenheft}
\author{Lisa Vogelsberg}
\date{29.11.2017}

\begin{document}
\tableofcontents
%Bei reports beginnt die Zählung von sections mit 0. Daher wurde hier chapter gewählt. Danach kann section gewählt werden, danach subsection und subsubsection
\chapter{Ausgangssituation}

%zwischen begin/end itemize können mit item stichpunkte gesetzt werden, auch Schachtelung möglich
\begin{itemize}
\item Dies ist ein Stichpunkt
\end{itemize}
\chapter{Zielsetzung und Produkteinsatz}
\section{Vision}
%LV: Ich krieg das nicht schön bzw. prägnant ausformuliert, deswegen erstmal so. Vielleicht kann das jemand besser ausdrücken.
Mitarbeiter von Unternehmen sollen mit Spaß und Unterhaltung dazu motiviert werden, sich an der betrieblichen Weiterbildung zu beteiligen. Wissen soll vermittelt werden, ohne dass es als Last empfunden wird. Im besten Fall wollen die Mitarbeiter sich an der Weiterbildung über das gegebene notwendige Maß hinaus beteiligen.
\section{Zielsetzung}
Die zu entwickelnde Software soll unter Verwendung verschiedener Techniken der Gamification die betriebliche Weiterbildung und den Wissensaustausch unterstützen.
Als Kernkomponente soll ein Quiz mit einer Frage, vier Auswahlmöglichkeiten und einer richtigen Antwort erstellt werden, an dem Gruppen von mindestens 5 Personen teilnehmen.
Gewinnen soll dabei aber nicht immer nur derjenige, der am meisten weiß, sondern derjenige, der Wissen, Strategie und Zufall für sich zu nutzen weiß.
\section{Produkteinsatz}
Zielgruppe des Produkts sind Arbeitnehmer in Unternehmen verschiedener Branchen, die sich während ihrer Arbeitszeit mit dem Quiz beschäftigen. Die Teilnahme erfolgt anonym dahingehend, dass nicht festgestellt werden kann, welche reale Person gerade an dem Quiz teilnimmt.
\chapter{Funktionale Anforderungen}
\section{Muss-Ziele}
\begin{itemize}
\item Dashboard zur Auswertung
\item Client muss bei jedem Login vom Server wiedererkannt werden
\item Datenbank zur internen Speicherung von Fragen und spielerbezogenen Daten
\item ACL zum Administrieren
\item Gamification Elemente:
\begin{itemize}
\item Badges
\item Ranking-System
\end{itemize}
\end{itemize}
\chapter{Nicht-funktionale Anforderungen}
\section{Muss-Ziele}
\begin{itemize}
\item 
\end{itemize}
\chapter{Qualitätsmatrix nach ISO 25010}
\begin{itemize}
\item
\end{itemize}
\chapter{Lieferumfang und Abnahmekriterium}
\begin{itemize}
\item
\end{itemize}
\chapter{Vorprojekt}
\begin{itemize}
\item
\end{itemize}
\chapter{Glossar}
\begin{itemize}
\item
\end{itemize}
\end{document}